\documentclass{article}
\author{John Dulin, jdd49}
\title{PHYS 250 Homework 3}

\begin{document}
\maketitle

\section*{Problem 1}
Consider $ f(x) = sin(x) $ and evaluate the derivative at $x = 0.9$

\subsection*{(i)}
Calculations for the relative error from using center differencing (specifically scipy.misc.derivative) at decreasing
values of $h$.

\begin{tabular}{cl | cl}
h & Relative error\\
\hline
$10^{-4}$ & $1.6666662894238016 * 10^{-7}$\\
$10^{-5}$ & $1.6667538460524156 * 10^{-9}$\\
$10^{-6}$ & $2.6271984587822317 * 10^{-11}$\\
$10^{-7}$ & $9.4488861179797823 * 10^{-12}$\\
$10^{-8}$ & $5.2636439651365663  * 10^{-10}$\\
$10^{-9}$ & $1.2596799159325656 * 10^{-9}$\\
\end{tabular}

\subsection*{(ii)}
From the equation for an error estimate,
$$ e(h) = \epsilon/h + (h^2/6)M $$
Derive the expression for h at which $e(h)$ is a minimum.

\vspace{60 mm}

\subsection*{(iii)}
For our $f(x)$ what is the maximum value M can have?  Use this to estimate the value of h for which $e(h)$ is a minimum, deriving M from the equation,
$$ M <= d^3f(x)/dx^3 $$

\vspace{60 mm}

\section*{Problem 2}
\subsection*{(i)}
Calculate $F_2(h)$ and generalize this to find the expression for $F_{j}(h)$
assuming that h gets halved to increase precision, $h→ h/2$.

\vspace{60 mm}

\subsection*{(ii)}
Looking at the expression you get compare this to what we found for the case of center
differencing. Explain based on this why you expect the algorithm using forward differencing
to converge more slowly.

\vspace{60 mm}

\subsection*{(iii)}
Using richardson\_center.py as a model, implement the previous part's algorithm.  Include the code or differences 
between the original code and yours in the solution.



\vspace{60 mm}

\section*{Problem 3}
Consider $ f(x) = 2^xsin(x) $ evaluated at $z = 1.05$ and $h = 0.4$

\subsection*{(i)}
Applying the product rule, we analytically find the derivative of $f(x)$ to be,
$$ f'(x) = 2^x(cos(x) + ln(2)sin(x)) $$
where $f'(1.05) = 2.30753$

\subsection*{(ii)}
The derivative $f'(z)$ is estimated to be $2.2751414822268563$ from Richardson Extrapolation of the center differencing
algorithm after 8 steps.

The error of the Richardson extrapolated center differencing algorithm is $|2.2751414822268563 - 2.30753| = 
0.0323885177731437$.  The error of the Richardson extrapolated forward differencing algorithm is $|2.29036535 - 2.30753| = 0.01716465.$

\subsection*{(iii)}
Determine the number of iterations, n, required for each algorithm to compute $f(z)$ to an
accuracy of $10^{−7}$.

\vspace{60 mm}

\subsection*{(iv)}
Is center differencing better because it requires fewer iterations?  
How many new function evaluations are required to calculate $F_1(h/2)$ in each case? Use this to find expressions for the
number of function evaluations needed to perform n iterations of Richardson extrapolation
for both algorithms.

\vspace{60 mm}

\subsection*{(v)}
Using your expressions from the previous part how many function evaluations are required
for the results from part (iii)?

\vspace{60 mm}

\subsection*{(vi)}
Find the minimum error and the number of iterations that give this minimum error for
estimating $f(z)$ for both algorithms.

\vspace{60 mm}

\subsection*{(vii)}
Again we see there is a minimum error? Why do the algorithms eventually fail? Mathematically we can take the 
limit $h$ approaches $0$, describe what happens numerically to prevent us from doing this.

\vspace{60 mm}

\end{document}
