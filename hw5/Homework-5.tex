\documentclass{article}
\author{John Dulin}
\title{PHYS 250 Homework 5}

\begin{document}
\maketitle

\section*{Problem 1}

\subsection*{(i)}
To derive the leap-frog approximation to $y_{n+1}$, we start with the center differencing formula for the first derivative,

$$ \frac{dy}{dt} = \frac{y(t + \Delta\/t) - y(t - \Delta\/t)}{2\Delta\/t} $$

From this, we simply solve for the next term in our solution function, $y(t + \Delta\/t)$, in terms of the 
known values, the differential equation $-dy/dt$ and the initial value $y(t - \Delta\/t)$.

$$ 2\Delta\/t\frac{dy}{dt} = y(t + \Delta\/t) - y(t - \Delta\/t)   $$
$$ y(t + \Delta\/t) = 2\Delta\/t\frac{dy}{dt} + y(t - \Delta\/t) $$ 

\subsection*{(ii)}

The error in $y_{n+1}$ can be found from the expression,

$$ y_{n+1} + \delta\/y_{n+1} = 2\Delta\/t\frac{dy}{dt} + y_{n-1}+  \delta\/y_{n-1} $$

and the fact that the error in each term is,

$$ \delta\/y_{n} = \beta\/y_{n-1} $$

So we have estimates for the error in 

\subsection*{(iii)}
We can determine whether the method is stable or unstable for different cases of differential equations by determinig whether the value of $\beta$ is below or above 1.0, respectively.

\section*{Problem 2}

\subsection*{(i)}
To find the maximum height of the projectile ignoring air resistance we define an equation for its position in time, assuming $y = 0$ is its starting height,

$$ y(t) = y_{0} + v_{0}t + \frac{1}{2}gt^{2} $$

Then we take its derivative with respect to $t$,

$$ \frac{dy}{dt} = v_{0} + gt $$

Setting this equal to $0$, we can calculate the exact time $t$ at which the proctile's velocity is zero and it reaches its apex,

$$ 0 = v_{0} + gt $$
$$ -gt = v_{0} $$
$$ t = \frac{v_{0}}{-g}$$

Knowing that $v_{0} = 8 m/s$ and $g = 9.8 m/s^2$, we calculate $t$ to be approximately 0.816 seconds.
So the maximum height of the projective is $y(0.816) = 3.262$ meters.

\subsection*{(ii)}
If air resistance is modeled as a force quadratic in velocity,

$$ F = -bv|v|$$

We can define the motion of the projectile as a system of first order differential equations by considering the acceleration as just the derivative of the unknown function of the velocity, $v$.  So from the equation,

$$ m\frac{d^{2}y}{dt^{2}} = -bv|v|$$

We have the set of differential equations,

$$ \frac{dy}{dt} = v  $$
$$ \frac{dv}{dt} = \frac{-bv|v|}{m} $$

\subsection*{(iii)}
Now we can numerically calculate our answer to question (i) with the assumption $ b = 0.002 $ kg/m,
So we use the scipy module 'odeint' like so, 
\begin{verbatim} 
integrate.odeint()
\end{verbatim}

\subsection*{(iv)}

\section*{Problem 3}
See attached.

\section*{Problem 4}

\subsection*{(i)}

\subsection*{(ii)}

\subsection*{(iii)}





























\end{document}
